\documentclass[11pt,english]{article}

%%%%%%%%%%%%%%%%%%%%%%%%%%%%%%%%%%%%%%%%%%%%%%%%%%%%%%%%%%%
% Packages
%%%%%%%%%%%%%%%%%%%%%%%%%%%%%%%%%%%%%%%%%%%%%%%%%%%%%%%%%%%

\usepackage{fullpage}
\usepackage[showframe=false,margin=1in]{geometry}
\usepackage{amsmath, amsthm, amssymb}
\usepackage[linesnumbered,ruled,vlined]{algorithm2e}
\usepackage{url}
\usepackage{fancyvrb}
\usepackage{listings}
\usepackage{graphicx}
\usepackage{listings}

\parindent=0pt
\setlength{\parskip}{1em}
\lstset{
  basicstyle=\fontsize{10}{13}\selectfont\ttfamily
}


\begin{document}

\title{CSC 4780/6780 \\
Mice Analysis}
\maketitle

Here is the contingency table:

\begin{tabular}{ r |  c  c | c }
 Gene & No Cancer & Has Cancer & \\
\hline
J & 93 & 37 & \textbf{130} \\
R & 20 & 1 & \textbf{21} \\
K & 34 & 5 & \textbf{39} \\
\hline
 & \textbf{147} & \textbf{43} & \textbf{190}
\end{tabular}


Here are the conditional proportions:

\begin{tabular}{ r |  c  c | c }
 Gene & No Cancer & Has Cancer & \\
\hline
J & 71.5\% & 28.5\% & \textbf{68.4\%} \\
R & 95.2\% & 4.8\% & \textbf{11.1\%} \\
K & 87.2\% & 12.8\% & \textbf{20.5\%} \\
\hline
 & \textbf{77.4\%} & \textbf{22.6\%} & \textbf{}
\end{tabular}



Here are the expected counts of the genes and cancer were independent:

\begin{tabular}{ r |  c  c | c }
 Gene & No Cancer & Has Cancer & \\
\hline
J & 100.6 & 29.4 & \textbf{130} \\
R & 16.2 & 4.8 & \textbf{21} \\
K & 30.2 & 8.8 & \textbf{39} \\
\hline
 & \textbf{77.4\%} & \textbf{22.6\%} & \textbf{}
\end{tabular}

$$X^2 =  8.4972$$

There are 2 degrees of freedom, so the p-value is given by:

$$p = 0.014284171167428195$$

It is unlikely that we would see these number if the genes and cancer were independent.

\end{document}
